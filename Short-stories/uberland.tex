\documentclass[12pt]{creativeWriting}

\title{Uberland}
\author{Eduardo Fernandes}

\begin{document}
\maketitle

Paulo Fura bufa de irritação. O quê? Oito e meia, já? Fônix! Desde às cinco que estava no volante de seu táxi e ainda não tinha apanhado nem um passageiro. Passa devagar pelo Pingo Doce do Arco do Cego a espera de que alguma velhota fizesse sinal mas nada. Fogo! Cadê as velhotas a sair do Pingo Doce? Será que ninguém mais apanha táxi hoje em dia? Culpa da Uber. Odeio Uber!

Está cheio de fome e o que tinha na carteira mal dava para comprar um café. Café? Mal dava para comprar um cigarro. Não conseguia parar de pensar em tabaco. Foda-se, onde é que eu tava com a cabeça quando decidi parar de fumar? Pelo amor da Santa! Desliza a mão pelo bolso da camisa para sentir o maço amassado que lá andava e suspirou, mas acaba por arremessar na boca duas pastilhas de nicotina. Maldito Uber!

Uma mão estendida lhe chama a atenção que, quase imediatamente, foca-se num rabo espetacular colado a uma saia minúscula com padrões de oncinha. É puta, de certeza. O que eu queria saber é como é que gaja aguenta andar com as pernas de fora num frio desses? Boa noite.

\speak[b]{Boa noite, amigo.} A jovem fala com um ar simpático \speak{Eu preciso ir até a Rotunda do Relógio. Tens ideia de quanto é que fica a viagem?}

\speak{Uns dez euros, mas coisa, menos coisa. Quinze no máximo}

\speak{Olha, a noite tem andado meio fraca...}

\speak{A quem o dizes.}

\speak{Pois... Enfim, a noite tem andado meio fraca. Será que a gente não conseguia fazer um acerto?}

\speak{Acerto?}

\speak{É. Tipo, eu faço-te um bico e tu levas-me lá.}

\speak{Um bico? Uhn... Diz-me uma cena. Tu gostas de foder?}

\speak{Uma foda já é muito cara para pagar uma viagenzinha até ali.}

\speak{Yá! Mas não foi isso que perguntei... Perguntei se tu gostas de foder?}

\speak{Se eu não gostasse, não era puta.}

\speak{Então fode-te!}

Fura arranca o táxi sem nem ouvir o que a prostituta começa a xingar. Bico é o tanas. Eu quero é comer, e não é puta. E fumar, já agora. Será que ninguém mais apanha táxi nessa cidade? Maldito Uber!

Enfia mais uma pastilha de nicotina na boca enquanto segue pela Almirante Reis e decide fazer uma tentativa no aeroporto. Depois de quase quinze minutos na fila da paragem de táxis entra um senhor de meia idade com uma maleta na mão.

\speak{Vou colocar na bagageira pro senhor vir mais confortável.}

\speak[b]{Precisa não, que é uma mala pequena.} O senhor responde com um sotaque brasileiro.

\speak{Não sei lá no Brasil mas aqui em Portugal é obrigatório que as malas sejam colocadas na bagageira}

\speak{Sério? Não sabia disso.}

\speak{Mas é. Tem taxista que não liga pra isso mas se a polícia perceber, a gente recebe uma coima.}

\speak{Uma o quê?}

\speak{Uma coima... Acho que no Brasil é multa. O senhor vai pra onde?}

\speak{Parque das Nações.}

\speak[b]{Ah tá.} Fura não disfarça a decepção pela viagem ser tão curta. \speak{O senhor está aqui a trabalho? É do Rio?}

\speak{Estou a trabalho sim. Não sou do Rio não, sou de Uberlândia}

Como é que é? Uber o quê? Odeio Uber! Fora do meu táxi. Não quero saber. Chama um Uber. E são cinco euros, três do mínimo mais o complemento de bagagem. Uberlândia, é o carago! Fura volta a seguir viagem vociferando consigo mesmo. Mete a mão no bolso e puxa do maço mas, num supremo esforço, volta a colocá-lo lá e enfia mais uma pastilha de nicotina no bolso. Vê uma mão estendida e pára.

Entra um jovem com a namorada e o táxi segue para Santos. Compraste tabaco? A jovem puxa um maço de Camel Activate da mala e dá ao namorado. Fura vê aquilo e não aguenta mais. Mete a mão no bolso da camisa, enfia um cigarro na boca e acende.

\speak[b]{O que é isso?} O jovem pergunta irritado. \speak{É proibido fumar no táxi.}

\speak{Mas vocês fumam!}

\speak{Não no táxi.}

\speak[b]{Cabrões do caralho} Fura reclama baixinho mas é ouvido pelo casal.

Como é que é? Exclama o rapaz. Quero o livro de reclamações. Pede a rapariga vermelha de raiva. Vão à merda os dois! E vão de Uber! Fura pára o táxi, põe os passageiros para fora e segue irritado.

Mais à frente, pára numa rulote e compra uma sandes com os cinco euros que recebera do brasileiro, devorando-a quase à velocidade da luz. Depois, puxa o maço de cigarros do bolso e fuma cinco de rajada, sentindo o prazer que só a nicotina pode proporcionar. Está ainda a sentir os efeitos da fumaça quando toca o telemóvel. É o chefe. Quer que Fura volte para a cooperativa imediatamente. Maldito Uber!

\rule[0.5ex]{\linewidth}{1pt}

Faz já meia hora que Fura está sentado junto ao Tejo e o sol está a nascer. Cabrões filhos da puta! Fura pensa no casalinho que fez queixas à cooperativa e amassa o papel com sua carta de demissão. Justa causa... Filhos da puta! Maldito Uber.

Agora o que é que eu vou fazer? Como é que eu vou conseguir emprego? Fura puxa o maço do bolso e fuma mais um cigarro. O sol tinge de vermelho as águas do Tejo. Fura joga longe a beata, balançando as pernas por cima do rio. Chega para frente e olha para baixo, quase como se fosse jogar-se. Fica ali por alguns segundos. Parado... Inerte...

Depois volta para trás e olha para o céu, flexionado o pescoço. Mete a mão no bolso e puxa o telemóvel, lembrando-se que há pouco tempo tinha feito um plano com acesso à Internet. Abre o Google e escreve: "como ser condutor da Uber".

\end{document}
