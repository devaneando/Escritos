\documentclass[portuguese,12pt,a4paper]{manuscript}

\usepackage{creativeWriting}

\begin{document}

\frontmatter

\title{(S)exChange Prejudice}
\author{Eduardo Fernandes}
\email{edu.fernandes.pt@gmail.com}
\makeFirstPage

\mainmatter

\speak{Oi gata, teu pai é pirata?}

\speak{Não... Não, não, não... Não. Não acredito que estou a ouvir um piropo desses. Ridículo!}

\speak{É... Pondo as coisas desse jeito...}

\speak{Tás a achar o quê? Que eu sou uma dessas brasileiras fáceis que vão pra cama com uma cena dessas?}

\speak{Ei! Que que cê tá a insinuar? Que toda brasileira é puta?}

\speak{Não! Claro que não, eu...}

\speak{Cara, eu odeio essa coisa de preconceito. É foda...}

\speak{Olha, desculpa-me... A sério, não quis ofender-te. É só que... Fônix! Aquele piropo foi do piorio.}

\speak{É... Não foi dos meus momentos mais brilhantes.}

\speak{Yah! Se ainda tivesse sido algo do tipo \textit{"tô a ver-te a folhear esse livro. Brutal! Tu curtes Bukowski? Eu curto bué aquela coisa misogênica dele.}}

\speak{Ok, tu também te amarras no Bukowski? Eu adoro aquela coisa misogênica dele.}

\speak{Assim não vale. Fui eu quem dei a dica. E depois, não podes vir assim com um piropo desses para uma rapariga que está numa livraria. Primeiro porque a última coisa que uma rapariga numa livraria está interessada é em curtir com alguém. Segundo porque não é só por eu seu uma rapariga que tenho que estar interessada em gajos.}

\speak{Então você é... Como é que se diz aqui em Portugal? Fufa?}

\speak{Ê pá, fufa é tão mau! Lésbica, se faz favor.  E yah, eu curto raparigas.}

\speak{Sério?! Foda-se! Agora eu fiquei até arrependido...}

\speak{Do piropo parvo? Devias estar mesmo.}

\speak{Não, é que... Na boa, você é mesmo muito gata... e super interessante... e lésbica... e gosta de mulheres... e até há seis meses atrás eu era mulher.}

\speak{Como é que é?!?!}

\speak{É. Eu vim pra Portugal pra fazer a cirurgia, sabe? Hoje faz exatamente seis meses que eu virei eu.}

\speak{Eu... não... acredito...}

\speak{Liga não que eu já estou acostumado com essa reação. Eu achava que ia tudo ia ser perfeito depois que virasse homem, mas não é. Não é fácil ser transexual. As pessoas simplesmente não aceitam.}

\speak{Não, não é isso, é que... Nossa, eu não acredito que isso está a acontecer! Olha, eu sei exatamente o que estás a passar porque, tipo... Não vais acreditar nisso mas eu também mudei de sexo. Eu nasci José Roberto, acreditas? Quem é que se chama José Roberto? Enfim, já lá vão três anos mas... Apresento-te Maria Rita de Castro.}

\speak{Cê tá de sacanagem!}

\speak{Não! Não estou não. É verdade.}

\speak{Mas tem que ser sacanagem porque... Você sabe como é que eu me chamava antes da cirurgia?}

\speak{Como?}

\speak{Maria Rita...}

\speak{Foda-se, no way! E agora?}

\speak{José Roberto. Sem sacanagem.}

\speak{Isso só pode ser o universo conspirando contra a gente. José Roberto? Foda-se!}

\speak{O universo ou o destino, sei lá... Mas que tem senso de humor, tem.}

\speak{Yah! E... Diz-me uma cena, o negócio aí em baixo? Ficou fixe? Funciona direitinho?}

\speak{Cara, aqui entre nós, eu não sei. Só agora que terminou o período de resguardo da cirurgia, sabe? Ainda não usei ele não.}

\speak{E tua namorada aguentou esse tempo todo sem pinar?}

\speak{Tenho namorada não.}

\speak{Sem namorada e virgem. Háháhá! Tu tens a noção de que viraste virgem de novo, não tens?}

\speak{Nunca tinha pensando nisso... Mas já que a gente tá a conversar, e a xaninha? Tá tudo nos trinques?}

\speak{Nada que um KY de vez em quando não resolva. Háháhá!}

\speak{Hêhêhê!}

\speak{Errr...}

\speak{Errr...}

\speak{Olha... Uhn... Tu te amarras no Bukowski? Eu adoro aquela coisa misogênica dele. Você não quer me dar teu celular pra gente se encontrar mais tarde, tomar um café e falar um pouco mais?}

\speak{Dá-me teu número que eu vou te dar um toque pra ficar gravado.}

\speak{Nove três meia nove dois sete...}

\speak{Olha, nem vale a pena. Eu moro aqui no edifício. Hoje não trabalho. Porque que a gente não sobe pra tomar um café lá em casa agora mesmo?}

\speak{Demorô! Vamo nessa.}

\speak{Ei, espera! Pensando bem, se calhar é melhor deixar para mais tarde.}

\speak{Porquê?}

\speak{Porque eu não quero que tu me aches fácil e que vou pra cama com qualquer um que acabei de conhecer.}

\speak{Eu nunca acharia isso, gata... a não ser que você fosse brasileira.}

\backmatter

\end{document}
