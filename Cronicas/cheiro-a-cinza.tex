\documentclass[12pt]{creativeWriting}

\title{Cheiro a cinza}
\author{Eduardo Fernandes}

\begin{document}
\maketitle

O azul do céu cheira a cinza, a ódio e a desprezo. Os corpos amarrados aos madeiros cheiram a fezes, a pele queimada e ao desespero de quem está a morrer nas fogueiras inquisitórias.

Padres erguem báculos decretando a purificação do pecado. Crucificai-o! Gritam os judeus e os italianos. Queimai-os! Gritam os Espanhóis e os portugueses. Cortai-lhes a cabeça! Gritam os franceses. O que gritaremos nós?
 
O negro da noite cheira a cinza, a gaz e a superioridade. Judeus magros, mortos, mártires cheiram a podridão e à neve branca, enquanto esperam para serem jogados nos fornos.

Comandantes e capitães erguem suas pistolas rifles enquanto bombas caem sobre Berlim. Em breve o cheiro de pólvora à queima roupa 

balas na própria cabeça porão um fim à história. Ninguém é superior a ninguém. Saberemos disso nós, europeus?

O verde da selva cheira a cinza, a pólvora e à infância perdida. Crianças-soldado matam-se umas às outras, empunhando metralhadoras refugadas do norte para proteger diamantes, petróleo o que quer que de lá se queira.

Os rugidos das feras imploram simplesmente serem deixadas em paz, sabendo que, onde quer que o homem branco chegue tudo acaba, desfazendo-se na alvura do nada, quebrada apenas pelo sangue. O que é que corre em nossas veias, europeus?

O bege do teclado cheira a azul, a vermelho, a amarelo e a verde. Rapazes imberbes atacam quem nem mesmo conhecem, e que odeiam, e que fazem *like*. 

\end{document}
