\documentclass{apaKadu}

\addbibresource{bibliografia.bib}

\title{A pirataria sob o olhar dos jornais online portugueses}

\author{Eduardo Fernandes}

\affiliation{Universidade Autónoma de Lisboa}

\abstract{Este artigo busca-se contextualizar a questão da \textit{pirataria}, dos direitos de autor e de cópia como controversa do ponto de vista da opinião pública, e ainda analisar forma a comunicação social tem abordado o tema no seu discurso}

\keywords{controvérsia, jornalismo, consenso, pirataria, direitos de autor}

\begin{document}

\maketitle

\section{Direito de autor}

A apropriação de ideias sem permissão é muito antiga, provavelmente tão antiga quanto o próprio homem. Claro
que, no início, o conhecimento no seu âmbito cultural era visto como algo que, à priori, existia para ser partilhado sem direitos de exclusividade. O sincretismo politeísta é exemplo dessa noção.

A carta magna de 1215, e as concepções modernas de direitos que dali se originaram, foram um dos fatores mais importantes a mudança da concepção de que o conhecimento não era meramente domínio público, mas que o autor de uma obra ou de uma ideia tinha direitos – ainda que a nível do reconhecimento – sobre ela, e que esses direitos precisavam ser respeitados.

A invenção a imprensa influencia diretamente a questão do direito do autor uma vez que permite a distribuição em massa de conteúdo, tornando um bem transacionável. Cria também a figura do distribuidor, que imprime as obras e as vende e, assim, o ato de copiar e redistribuir uma obra não passa a afetar não somente o autor, mas também queles que detêm o direito de distribuí-la.

Uma das primeiras soluções encontradas para proteger o direito de cópia foi controlar o meio de reprodução, o que leva ao terceiro ponto relevante para este artigo: a neutralidade do meio.

\printbibliography

\end{document}
