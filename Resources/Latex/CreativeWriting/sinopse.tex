\documentclass[a4paper,12pt]{character}
\usepackage[T1]{fontenc}
\usepackage[utf8]{inputenc}
\usepackage{lmodern}

\title{O hospita psiquiátrico}
\author{Eduardo Fernandes}

\begin{document}

\maketitle

\section{Prólogo}

Um homem, já com seus cinquenta anos de idade, está acorrentado a um cadeirão antigo, numa sala sem janelas. Dois homens muito bem vestidos entram porta adentro, carregando um terceiro com as mãos atadas atrás das costas, e jogam-no no chão.

A medida que a vítima se arrasta e se encosta a uma parede, ficando de frente para seus algozes. Um deles tira uma faca de dentro do casaco \speak{Vocês os dois já deram demasiado trabalho...}, segura-lhe a cabeça pelos cabelos e lhe degola o pescoço com um gesto rápido. O moribundo começa a tossir, engasgando-se com o sangue a medida que se contorce. A voz de fundo do narrador começa a falar.

\speak[b]{Alguém já disse que tudo na história da humanidade acontece pela sede do poder, do dinheiro ou do sexo. No meu caso foi pelos três, é só uma questão de escolher como começar. Pelo poder?} Vê-se imagens de um nobre, provavelmente do século XVIII, a olha para navios mercantes no Rio Tejo. \speak[b]{Talvez pelo dinheiro...} Vê-se imagens de um delegado de polícia a mandar espancar um homem numa favela carioca. \speak[b]{Não. É sempre melhor começar pelo sexo...} Vê-se as imagens do moribundo, agora bem vestido a conversar com outro num campus universitário.

\section{Ato I}

Veem-se imagens de Alonso na faculdade, contextualizando-o na sua vida normal. Marca com uns amigos de ir à uma discoteca. Em meio à pista de dança, está a conversar com uma rapariga. Ela fala qualquer coisa sobre estar precisando de cheirar. Alonso diz que conhece o chefe do Morro de Dona Marta. Vão os dois para lá.

Alonso dá um abraço no traficante e começa uma conversa fiada. Chega um outro traficante e fala qualquer coisa no ouvido do chefe, que fica sério de repente e manda Alonso se esconder com a rapariga. Chega um conhecido delegado de polícia, exigindo dinheiro do traficante. Tem início uma discussão, e o delegado mata o traficante à queima-roupa. A rapariga grita, e o delegado descarrega a pistola na porta, matando-a. Alonso é atingido no ombro, mas tem sangue frio para pegar uma barra de ferro caída por ali. Quando o delegado vai recarregar a arma, ele dá com ela em sua cabeça, nocauteando-o.

Vê, largado numa mesa, o dinheiro que o traficante ia entregar ao delegado. Num impulso agarra-o e foge. Ferido, vai ter à casa de uma ex-namorada que lhe presta primeiros socorros. É então que vê seu rosto nos jornais e percebe que foi acusado não só de matar a rapariga e o traficante, como de tentar matar o delegado. Desesperado, pega o carro da ex e foge.

Vê imagens dela a largar o carro e comprar um bilhete de autocarro para o Paraguai, apanhando de lá um avião para Lisboa.

\section(Ato II)

Durante a viagem, ele está febril, delirando também por causa dos analgésicos que a ex lhe deu. Em meio à febre, tem uma conversa muito estranha sobre a morte com um senhor que lhe diz que morreu naquele mesmo avião, três anos antes. Ele diz que era um bem sucedido empresário, e que tinha esquemas secretos envolvendo contrabando com o SIS e com o serviços de estrangeiros e fronteiras.

Ao chegar no aeroporto, Alonso é abordado pelos serviços de estranheiros e fronteiras que descobrem o dinheiro que tem. Alonso sabe que se se for mandado de volta para o Brasil, está morto e decide jogar pelo desespero. Lembra da conversa que teve com o fantasma e manda chamar o homem que o velho falou, usando parte do teor da conversa para safar-se. Sai do aeroporto e hospeda-se numa pensão barata no Cais do Sodré.

A casa de banho é comum, e Alonso tentar tomar banho para limpar o ferimento infeccionado mas a febre e a dor acabam por levar a melhor e ele desmaia o chuveiro. Acorda no quarto, com uma prostituta já com seus cinquenta anos que lhe limpou os ferimentos. Trocam algumas palavras, e ela diz para ele dormir que depois lhe traá comida.

Os próximos dias são uma luta constante contra a febre e contra a agorafobia, já que Alonso está de tal maneira tomado pelo menos que se recusa a sair do quarto. A medida que vai ficando amigo de Marta, a prostituta, vai se aventurando pelas redondezas e conhecendo a cidade. Uma noite, encontra o quarto aberto e dá de cara com dois assaltantes. Pensando que eles são assassinos enviados para matá-lo, Alonso luta mas acaba espancado. A polícia toma nota da ocorrência e avisa o consulado que, por sua vez, avisa o Brasil.

O cenário muda para um bar carioca, onde se vê o delegado a conversar com um repórter e lhe dar em off o \textit{furo} de que será o próximo secretário de segurança pública do Rio. Aparece um homem com cara de mau. O reporter sai. O homem diz que descobriu onde Alonso está, e o delegando manda que ele vá imediatamente matá-lo.

Alonso está sem dinheiro e foi despejado da pensão. Marta deixaa que ele fique alguns dias em sua casa, mas ele precisa conseguir emprego.



\end{document}
