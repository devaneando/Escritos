\documentclass{creativeWriting}

\usepackage{epigraph}

\author{Eduardo Fernandes}
\title{Olho nos olhos}

\begin{document}

\maketitle

\section{Ele}

Olho bem fundo para aqueles olhos azuis, carregados de lágrimas. Vem-me à cabeça uma música do Chico Buarque.

\textit{Quero ver o que você faz ao sentir...}

Não, não quero ver nada. Não quero estar aqui nem ter esta conversa. Não quero lembrar de uma estupida música sobre um amor que se acabou até porque é isso mesmo que estou agora a fazer. Destruir um amor, uma vida, um futuro.

Não consigo desviar o olhar daquele cabelo louro, daquela pele branca, daqueles olhos azuis que em breve nunca mais olharei. Uma pontada de dor atravessa meu coração ao perceber que estou a terminar com tudo. Tudo!

\textit{Quase enlouqueci mas depois, como era de costume, obedeci...}

Viro-me para o marido. Odeia-me. Depois do que acabei de dizer, qualquer marido odiaria-me. Baixo os olhos e a única coisa que encontro é uma caneta largada na mesa.

Bic. Leio aquela marca vezes e vezes sem conta como se fosse um mantra, com o interesse de quem está mergulhado no melhor livro alguma vez já escrito. Bic. Bic. Bic. Qualquer coisa é melhor que voltar a olhar para aqueles olhos azuis, para aqueles lábios. O que me responderá? O que dirá para o marido? É o fim.

\textit{Quando você me deixou, meu bem, me disse pra ser feliz e passar bem...}

Tenho que ser firme. Tenho que ser forte. Alguém nesta sala tem que ser. Viro-me novamente para o marido, depois para ela. Estendo-lhe a mão para tocar em seu ombro e ela dá-me um tapa. Maldito!

Nossos olhares cruzam-se uma última vez. Levanta-se de um salto, com tamanha violência que derruba a cadeira. O choro já não pode ser contido e ela começa a correr porta a fora.

\textit{Quando talvez precisar de mim, você sabe que a casa é sempre sua, venha sim}

 O marido olha-me. Sua expressão já não é de raiva, mas de tristeza. Estendo-lhe a mão e, agora, é a sua vez de olhar para baixo, virar e sair atrás daquela que ama.
 
 É o fim.

\section{Ela}

Não consigo conter as lágrimas. Não consigo desviar os olhos daqueles olhos castanhos, daquela bata branca, daquele estetoscópio.

\textit{Quando tudo está perdido, sempre existe um caminho...}

Não consigo fazer nada. Não consigo nem mesmo sentir. O que será agora do meu marido? Penso em todas as brigas parvas que tivemos. Quanta inutilidade, meu Deus! Discutir por causa de meias e de migalhas de pão. Está tudo acabado. 

\end{document}
